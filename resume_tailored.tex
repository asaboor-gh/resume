\documentclass[letter,11pt]{article}
\usepackage[left=0.8in,right=0.8in,top=0.8in,bottom=0.8in]{geometry}
\usepackage{titlesec}
\usepackage{xcolor}
\usepackage{fontawesome}
\usepackage{hyperref}

\hypersetup{
    colorlinks=true,
    linkcolor=blue!80!black,      
    urlcolor=blue!80!black,
}

\urlstyle{same}

\titleformat{\section}{\large\bfseries\color{blue!60!black}}{}{0em}{} 

\begin{document}

\begin{minipage}{0.65\textwidth}
    \raggedright
    {\Large \textbf{ABDUL SABOOR}} \\ 
    Department of Physics and Astronomy, University of Delaware, Newark, DE 19716 
\end{minipage}
\hfill
\begin{minipage}{0.3\textwidth}
    \raggedright
    \faGithub \quad \href{https://github.com/asaboor-gh}{asaboor-gh}\\
    \faLinkedin \quad \href{https://linkedin.com/in/asaboor-in}{asaboor-in} \\
    \faPhone \quad \href{tel:+13027227047}{(302) 722-7047} \\
    \faEnvelope \quad \href{mailto:asaboor@udel.edu}{asaboor@udel.edu}
\end{minipage}
\vspace{2mm}
\hrule
\vspace{4mm} 

\noindent
\textbf{Summary:} Ph.D. Physicist specializing in the computational modeling and simulation of advanced semiconductor materials for next-generation computer architectures and novel memory systems. Proven expertise in developing and applying computational algorithms to solve complex materials science problems, directly informing hardware design. A highly motivated researcher with a strong record of publication, open-source software development, and collaboration with interdisciplinary teams.

\section{Education}
\begin{tabular}{p{3.25cm} p{12cm}}
    \textbf{2025 (Tentaive)} & Ph.D. in Physics, University of Delaware, Newark, DE \\
    \textbf{2025} & M.S. in Physics(en route to Ph.D.), University of Delaware, Newark, DE \\
    \textbf{2017} & M.Phil. in Physics, Quaid-i-Azam University, Islamabad \\
    \textbf{2015} & M.Sc. in Physics, Quaid-i-Azam University, Islamabad \\
    \textbf{2012} & B.Sc. in Mathematics \& Physics, University of Azad Jammu \& Kashmir \\
\end{tabular}

\section{Core Competencies}
\begin{itemize}
    \item \textbf{Computational Modeling \& Simulation:} Expertise in simulating material properties using Density Functional Theory (DFT) to design and engineer electronic structures for novel memory systems and non-Von Neumann computing architectures.
    \item \textbf{Algorithm \& Compiler Development:} Strong skills in Python for developing computational algorithms, automating complex simulations, and creating scientific software packages that interface with low-level hardware models.
    \item \textbf{Computer Architecture \& Materials Science:} In-depth knowledge of how semiconductor physics, including strain engineering in III-V alloys and 2D materials, directly impacts the performance and design of next-generation computing hardware.
    \item \textbf{Collaboration \& Mentorship:} Proven ability to work with interdisciplinary research teams, mentor graduate and undergraduate students, and collaborate on research proposals with academic and industry partners.
\end{itemize}

\section{Research Experience}
\begin{itemize}
    \item Led large-scale DFT simulations to model the electronic and structural properties of novel semiconductor alloys, directly supporting the design of advanced memory systems and non-Von Neumann computing hardware.
    \item Engineered material properties, such as band-gaps and strain effects, in III-V alloys and 2D materials, providing foundational research for next-generation electronic devices and informing compiler-level optimizations.
    \item Authored and co-authored research papers for high-impact peer-reviewed journals, including \textit{Nature Nanotechnology}, and prepared research for publication.
    \item Collaborated with internal and external academic and industry partners to validate computational models and guide material synthesis, demonstrating strong teamwork and communication skills.
    \item Mentored graduate and undergraduate students on various research projects, fostering a collaborative and productive team environment.
\end{itemize}

\section{Teaching Experience}
\begin{itemize}
    \item \textbf{Physics Teaching Assistant at Quaid-i-Azam University:} (2017) \\ 
        Assisted in teaching, grading and laboratory sessions for undergraduate students in Spring 2017.
    \item \textbf{Introductory Physics I \& II (PHYS 201, PHYS 202):} Teacher (2018-2022), TA (2019-2021)
    \item \textbf{Fundamentals of Physics I \& II (PHYS 207, PHYS 208):} TA (2019-2025)
    \item \textbf{Fundamentals of Physics Laboratory II (PHYS 228):} Teacher (2022)
    \item \textbf{Fundamentals of Physics with Biomedical Applications II (PHYS 204):} TA (2022-2024)
    \item \textbf{Physics Online Lab Development:} Teacher (2020)
    \item \textbf{Physics Help Center:} TA (2018-2025)
\end{itemize}

\section{Technical Skills}
\begin{itemize}
    \item \textbf{Programming:} Python, MATLAB, Mathematica, PowerShell, Julia (learning)
    \item \textbf{Simulation Software:} VASP, Quantum ESPRESSO, LAMMPS
    \item \textbf{Development Tools:} Git, Jupyter, Linux, \LaTeX
    \item \textbf{Open Source Development:} Creator of \href{https://github.com/asaboor-gh/ipyvasp}{ipyvasp} for VASP automation and \href{https://github.com/asaboor-gh/ipyslides}{ipyslides} for interactive presentations.
\end{itemize}

\section{Selected Publications}
\begin{itemize}
    \item S. Nair, \textbf{A. Saboor}, et al., "Engineering metal oxidation using epitaxial strain," \href{https://www.nature.com/articles/s41565-023-01397-0}{\textit{Nat. Nanotechnol.}, (2023)}
    \item \textbf{A. Saboor}, S. Khalid, A. Janotti, "Band-gap reduction and band alignments of dilute bismide III-V alloys," \href{https://arxiv.org/abs/2411.19257}{\textit{arXiv:2411.19257} [cond-mat]}
\end{itemize}

\section{Conference Presentations}
\begin{itemize}
    \item The 67$^{th}$ Electronic Materials Conference, Duke University NC, 2025
    \item PyCon US, Pittsburgh, 2025
    \item American Physical Society (APS) March Meeting, Minneapolis, 2024
    \item American Physical Society (APS) March Meeting, Las Vegas, 2023
\end{itemize}

\section{References}
Available upon request.

\end{document}