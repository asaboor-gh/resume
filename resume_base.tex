\documentclass[letter,11pt]{article}
\usepackage[left=0.8in,right=0.8in,top=0.8in,bottom=0.8in]{geometry}
\usepackage{titlesec}
\usepackage{xcolor}
\usepackage{fontawesome}
\usepackage{hyperref}

\hypersetup{
    colorlinks=true,
    linkcolor=blue!80!black,      
    urlcolor=blue!80!black,
}

\urlstyle{same}

\titleformat{\section}{\large\bfseries\color{blue!60!black}}{}{0em}{} 

\begin{document}

\begin{minipage}{0.65\textwidth}
    \raggedright
    {\Large \textbf{ABDUL SABOOR}} \\ 
    Department of Physics and Astronomy, University of Delaware, Newark, DE 19716 
\end{minipage}
\hfill
\begin{minipage}{0.3\textwidth}
    \raggedright
    \faGithub \quad \href{https://github.com/asaboor-gh}{asaboor-gh}\\
    \faLinkedin \quad \href{https://linkedin.com/in/asaboor-in}{asaboor-in} \\
    \faPhone \quad \href{tel:+13027227047}{(302) 722-7047} \\
    \faEnvelope \quad \href{mailto:asaboor@udel.edu}{asaboor@udel.edu}
\end{minipage}
\vspace{2mm}
\hrule
\vspace{4mm} 

\noindent
\textbf{Summary:} Ph.D. Physicist specializing in the computational modeling and simulation of advanced semiconductor materials for next-generation computer architectures and novel memory systems. Proven expertise in developing and applying computational algorithms to solve complex materials science problems, directly informing hardware design. A highly motivated researcher with a strong record of publication, open-source software development, and collaboration with interdisciplinary teams.

\section{Education}
\begin{tabular}{p{3.25cm} p{12cm}}
    \textbf{2025 (Expected)} & Ph.D. in Physics, University of Delaware, Newark, DE \\
    \textbf{2025} & M.S. in Physics(en route to Ph.D.), University of Delaware, Newark, DE \\
    \textbf{2017} & M.Phil. in Physics, Quaid-i-Azam University, Islamabad \\
    \textbf{2015} & M.Sc. in Physics, Quaid-i-Azam University, Islamabad \\
    \textbf{2012} & B.Sc. in Mathematics \& Physics, University of Azad Jammu \& Kashmir \\
\end{tabular}

\section{Research Experience}
\begin{itemize}
    \item Led large-scale DFT simulations to model the electronic and structural properties of novel semiconductor alloys, directly supporting the design of advanced memory systems and non-Von Neumann computing hardware.
    \item Engineered material properties, such as band-gaps and strain effects, in III-V alloys and 2D materials, providing foundational research for next-generation electronic devices and informing compiler-level optimizations.
    \item Authored and co-authored research papers for high-impact peer-reviewed journals, including \textit{Nature Nanotechnology}, and prepared research for publication.
    \item Mentored fellow graduate students with coding for analysis in their research, fostering a collaborative and productive team environment.
\end{itemize}

\section{Teaching Experience}
\begin{itemize}
    \item \textbf{Physics Teaching Assistant at Quaid-i-Azam University:} (2017) \\ 
        Assisted in teaching, grading and laboratory sessions for undergraduate students in Spring 2017.
    \item \textbf{Introductory Physics I \& II (PHYS 201, PHYS 202):} (2018-2022) \\
        Supervised undergraduate laboratory sessions, graded assignments, and provided academic support.
    \item \textbf{Fundamentals of Physics I \& II (PHYS 207, PHYS 208):} (2019-2023) \\
        Supervised laboratory sessions, graded assignments, and provided academic support.
    \item \textbf{Fundamentals of Physics with Biomedical Applications II (PHYS 204):} (2022-2024) \\
        Supervised laboratory sessions, graded assignments, and provided academic support.
    \item \textbf{Physics Online Lab Development:} (2020) \\
        Developed online laboratory coontent for undergraduate physics in collaboration with faculty and TAs.
    \item \textbf{Fundamentals of Physics Laboratory II (PHYS 228):} (2022-2025) \\
        Supervised discussions and laboratory sessions, graded assignments, and provided academic support.
    \item \textbf{Physics Help Center:} TA (2018-2025) \\
        Provided academic support to undergraduate students in introductory physics courses, assisting with problem-solving and conceptual understanding.
\end{itemize}

\section{Technical Skills}
\begin{itemize}
    \item \textbf{Programming Languages:} Python, MATLAB, Mathematica, PowerShell, Julia (learning)
    \item \textbf{Scientific Software:} \href{https://www.vasp.at/}{VASP}, \href{https://www.quantum-espresso.org/}{Quantum ESPRESSO}, \href{https://wiki.fysik.dtu.dk/ase/}{ASE}, \href{https://nanohub.org}{nanohub}, \href{https://kwant-project.org}{Kwant}, \href{https://axelvandewalle.github.io/www-avdw/atat/}{ATAT}
    \item \textbf{Developer Tools:} Git, Jupyter, VS Code, Linux, Conda
    \item \textbf{Open Source Projects Authored:}
    \begin{itemize}
        \item \href{https://github.com/asaboor-gh/ipyvasp}{ipyvasp}, a Python package for automating and analyzing VASP simulations.
        \item \href{https://github.com/asaboor-gh/ipyslides}{ipyslides}, a tool for creating interactive presentations within Jupyter Notebooks.
        \item \href{https://github.com/asaboor-gh/einteract}{einteract}, a library for building interactive dashboards in Jupyter notebooks.
    \end{itemize}
\end{itemize}

\section{Publications}
\begin{itemize}
    \item S. Nair, \textbf{A. Saboor}, et al.,``Engineering metal oxidation using epitaxial strain," \href{https://www.nature.com/articles/s41565-023-01397-0}{\textit{Nat. Nanotechnol.}, (2023)}
    \item \textbf{A. Saboor}, S. Khalid, A. Janotti, ``Band-gap reduction and band alignments of dilute bismide III-V alloys," \href{https://arxiv.org/abs/2411.19257}{\textit{arXiv:2411.19257} [cond-mat]} (2024)
    \item \textbf{A. Saboor}, ``ipyvasp: A Python Package for Interactive Analysis and Visualization of VASP Data". Zenodo, \href{https://zenodo.org/records/15482349}{\textit doi: 10.5281/zenodo.15482349} (2025)
    \item \textbf{A. Saboor}, ``ipyslides: A Python Framework for Creating Interactive Presentations in Jupyter Notebooks", \href{https://zenodo.org/records/15482496}{\textit doi: 10.5281/zenodo.15482496} (2025)
    \item I. Evangelista, I. Chatratin, R. Hu, D. Q. Ho, \textbf{A. Saboor}, M. Zubair, S. Khalid, I. Fampiou, and A. Janotti. ``Effects of uniaxial stress and biaxial strain on the electronic properties of
        monolayer transition-metal dichalcogenides." (submission ready)
    \item \textbf{A. Saboor}, R. Hu, and A. Janotti. ``Electronic properties of InAlAs and InGaAs alloys
containing a few percent of Bi." (in progress)
    \item R. Hu, W. Acuna, \textbf{A. Saboor}, D. Q. Ho, J. Zide, G. W. Bryant, and A. Janotti. ``Rare-earth
monopnictides nanoparticles embedded in bismide III-V alloys for THz devices." (in
progress)
\end{itemize}

\section{Conference Presentations}
\begin{itemize}
    \item The 67$^{th}$ Electronic Materials Conference, Duke University NC, (2025) \\
      Presented: ``Electronic properties of InAlAs and InGaAs alloys containing a few percent of Bi"
    \item The Franklin Institute Awards Symposium, Temple University, (2025)
    \item PyCon US, Pittsburgh, (2025)
    \item American Physical Society (APS) March Meeting, Minneapolis, (2024) \\
      Presented (by advisor): ``Electronic properties of InAlAs and InGaAs alloys containing a few percent of Bi"
    \item American Physical Society (APS) March Meeting, Las Vegas, (2023) \\
      Presented: ``Electronic structure and band alignment of dilute III-V$_{1-x}$Bi$_x$ alloys"
    \item SCAN Workshop, Temple University, (2019)
\end{itemize}

\section{References}
\textbf{Prof. Anderson Janotti} \\
Department of Material Science and Engineering, University of Delaware \\
Email: \href{mailto:janotti@udel.edu}{janotti@udel.edu} \\

\noindent\textbf{John Shaw} \\
Assistant Professor and Lab Manager \\
Department of Physics and Astronomy, University of Delaware \\
Email: \href{mailto:jdshaw@udel.edu}{jdshaw@udel.edu}
\end{document}