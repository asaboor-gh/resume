\documentclass[letter,11pt]{article}
\usepackage[left=0.8in,right=0.8in,top=0.8in,bottom=0.8in]{geometry}
\usepackage{titlesec}
\usepackage{xcolor}
\usepackage{fontawesome}
\usepackage{hyperref}

\hypersetup{
    colorlinks=true,
    linkcolor=blue!60!black,      
    urlcolor=blue!60!black,
    }

\urlstyle{same}

\titleformat{\section}{\large\bfseries\color{blue!50!black}}{}{0em}{} 

\begin{document}


\begin{minipage}{0.5\textwidth}
    \raggedright
    {\Large \textbf{ABDUL SABOOR}} \\ Department of Physics and Astronomy, University of Delaware, Newark, DE 19716 
\end{minipage}
\hfill
\begin{minipage}{0.3\textwidth}
    \raggedright
    \faGithub \quad \href{https://github.com/asaboor-gh}{asaboor-gh}\\
    \faLinkedin \quad \href{https://linkedin.com/in/asaboor-in}{asaboor-in} \\
    \faPhone \quad \href{tel:+13027227047}{(302) 722-7047} \\
    \faEnvelope \quad \href{mailto:asaboor@udel.edu}{asaboor@udel.edu}
\end{minipage}
\vspace{2mm}
\hrule
\vspace{5mm}

\section{Research Interests}
Density functional theory (DFT) applications, electronic structure tuning in III-V semiconductors, strain effects in materials, and theoretical modeling of material properties.

\section{Education}
\begin{tabular}{p{3cm} p{12cm}}
    \textbf{2017--Present} & Ph.D. in Physics, University of Delaware \\
    \textbf{2017} & M.Phil. in Physics, Quaid-i-Azam University \\
    \textbf{2015} & M.Sc. in Physics, Quaid-i-Azam University \\
    \textbf{2012} & B.Sc. in Mathematics \& Physics, University of Azad Jammu \& Kashmir \\
\end{tabular}

\section{Research Experience}

\begin{itemize}
\item Investigated band gap engineering in III-V alloys incorporating Bi for mid-infrared applications.
\item Analyzed rare-earth monopnictide nanoparticles embedded in bismide III-V alloys.

\item Studied epitaxial strain effects on III-V materials and transition-metal dichalcogenides.
\item Examined metal oxidation in IrO$_2$ using density functional theory.
\end{itemize}

\section{Technical \& Computational Skills}
\begin{itemize}
\item Developed \texttt{ipyvasp}, a Python package for processing VASP calculations.
\item Proficient in Python, PowerShell, Bash, MATLAB, and Mathematica; expanding expertise in Julia.
\item Extensive experience with computational simulations and data analysis.
\end{itemize}

\section{Publications}
\begin{itemize}
    \item S. Nair, et al., "Engineering metal oxidation using epitaxial strain," \textit{Nat. Nanotechnol.}, 2023.
    \item A. Saboor, S. Khalid, A. Janotti, "Band-gap reduction and band alignments of dilute bismide III-V alloys," \textit{Phys. Rev. Mater.} (Submitted, 2024).
\end{itemize}

\section{Honors \& Awards}
\begin{itemize}
\item Doctoral scholarship, UNAM, Mexico (2015).
\item World Federation of Scientists (WFS) scholarship, Switzerland (2016).
\end{itemize}

\section{Conference Participations}
\begin{itemize}
\item The 67$^{th}$ Electronic Materials Conference, Duke University NC 2025
\item PyCon US, Pittsburgh 2025
\item The Franklin Institute Awards Symposium, Temple University, 2025
\item SCAN Workshop, Temple University, 2019.
\item APS March Meeting, 2023.
\item APS March Meeting, 2024 (presented by advisor).
\end{itemize}

\section{References}
\textbf{Prof. Anderson Janotti} \\
Department of Material Science and Engineering, University of Delaware \\
Email: \texttt{janotti@udel.edu}

\end{document}
