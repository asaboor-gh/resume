% filepath: resume_industry_test_engineer.tex
\documentclass[letter,11pt]{article}
\usepackage[left=0.8in,right=0.8in,top=0.8in,bottom=0.8in]{geometry}
\usepackage{titlesec}
\usepackage{xcolor}
\usepackage{fontawesome}
\usepackage{hyperref}

\hypersetup{
    colorlinks=true,
    linkcolor=blue!80!black,      
    urlcolor=blue!80!black,
}

\urlstyle{same}

\titleformat{\section}{\large\bfseries\color{blue!60!black}}{}{0em}{} 

\begin{document}

\begin{minipage}{0.65\textwidth}
    \raggedright
    {\Large \textbf{ABDUL SABOOR}} \\ 
    Department of Physics and Astronomy, University of Delaware, Newark, DE 19716 
\end{minipage}
\hfill
\begin{minipage}{0.3\textwidth}
    \raggedright
    \faGithub \quad \href{https://github.com/asaboor-gh}{asaboor-gh}\\
    \faLinkedin \quad \href{https://linkedin.com/in/asaboor-in}{asaboor-in} \\
    \faPhone \quad \href{tel:+13027227047}{(302) 722-7047} \\
    \faEnvelope \quad \href{mailto:asaboor@udel.edu}{asaboor@udel.edu}
\end{minipage}
\vspace{2mm}
\hrule
\vspace{4mm} 

\noindent
\textbf{Summary:} Detail-oriented Ph.D. Physicist with a strong background in semiconductor physics and computational modeling. Expertise in developing Python-based tools for data acquisition and analysis, designing complex test schemes, and troubleshooting hardware-software interactions. A resourceful problem-solver with proven ability to interpret technical specifications, document results thoroughly, and collaborate effectively with cross-functional teams.

\section{Education}
\begin{tabular}{p{3.25cm} p{12cm}}
    \textbf{2025 (Expected)} & Ph.D. in Physics, University of Delaware, Newark, DE \\
    \textbf{2025} & M.S. in Physics, University of Delaware, Newark, DE \\
    \textbf{2017} & M.Phil. in Physics, Quaid-i-Azam University, Islamabad \\
    \textbf{2015} & M.Sc. in Physics, Quaid-i-Azam University, Islamabad \\
    \textbf{2012} & B.Sc. in Mathematics \& Physics, University of Azad Jammu \& Kashmir \\
\end{tabular}

\section{Relevant Experience}
\begin{itemize}
    \item \textbf{Computational Research \& Development} (2018-Present)
        \begin{itemize}
            \item Designed and implemented comprehensive simulation schemes to test and validate the performance of novel semiconductor materials, interpreting complex physical specifications to define test parameters.
            \item Analyzed large datasets from simulations to assess material performance, identify performance trends, and flag deviations from theoretical models.
            \item Authored detailed technical documentation and reports on simulation setups, procedures, and results, supporting peer-reviewed publications.
            \item Collaborated with experimental teams to troubleshoot discrepancies between simulated and real-world performance, contributing to hardware-level solutions.
        \end{itemize}
    \vspace{2mm}
    \item \textbf{Technical Instruction \& Training} (2018-2025)
        \begin{itemize}
            \item Trained and mentored undergraduate engineering students in laboratory settings, including the use of test instruments and data acquisition software such as \textbf{LabView} for the Fundamentals of Physics Laboratory II (PHYS 228).
            \item Developed and delivered technical content for a variety of physics courses, effectively communicating complex topics to diverse audiences.
            \item Managed multiple lab sections and projects simultaneously, ensuring all objectives were met on schedule.
        \end{itemize}
\end{itemize}

\section{Technical Skills}
\begin{itemize}
    \item \textbf{Programming Languages:} Python (Expert), MATLAB, Mathematica
    \item \textbf{Test \& Lab Software:} LabView (Familiar), VASP, Quantum ESPRESSO, ASE
    \item \textbf{Developer Tools:} Git, VS Code, Jupyter, pytest, pip, Conda, Linux
    \item \textbf{Core Competencies:} Test Development, Data Analysis, Technical Documentation, Resourceful Problem-Solving, Project Management, Technical Training
\end{itemize}

\section{Authored Open Source Software}
\begin{itemize}
    \item \href{https://github.com/asaboor-gh/ipyvasp}{ipyvasp}: A Python package for automating and analyzing VASP simulations, demonstrating skills in creating robust test and analysis frameworks.
    \item \href{https://github.com/asaboor-gh/ipyslides}{ipyslides}: A tool for creating interactive presentations within Jupyter Notebooks, showcasing the ability to develop clear and effective communication tools.
    \item \href{https://github.com/asaboor-gh/einteract}{einteract}: A library for building interactive dashboards in Jupyter, highlighting skills in user interface design for data analysis.
\end{itemize}

\section{Selected Publications \& Presentations}
\begin{itemize}
    \item S. Nair, \textbf{A. Saboor}, et al., ``Engineering metal oxidation using epitaxial strain," \href{https://www.nature.com/articles/s41565-023-01397-0}{\textit{Nat. Nanotechnol.}, (2023)}
    \item \textbf{A. Saboor}, ``ipyvasp: A Python Package for Interactive Analysis and Visualization of VASP Data," Zenodo, \href{https://zenodo.org/records/15482349}{\textit doi: 10.5281/zenodo.15482349} (2025)
    \item American Physical Society (APS) March Meeting, Las Vegas, (2023) \\
      Presented: ``Electronic structure and band alignment of dilute III-V$_{1-x}$Bi$_x$ alloys"
\end{itemize}

\section{References}
Available upon request.
\end{document}