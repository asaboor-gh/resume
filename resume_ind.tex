\documentclass[letter,11pt]{article}
\usepackage[left=0.8in,right=0.8in,top=0.8in,bottom=0.8in]{geometry}
\usepackage{titlesec}
\usepackage{xcolor}
\usepackage{fontawesome}
\usepackage{hyperref}

\hypersetup{
    colorlinks=true,
    linkcolor=blue!80!black,      
    urlcolor=blue!80!black,
    }

\urlstyle{same}

\titleformat{\section}{\large\bfseries\color{blue!60!black}}{}{0em}{} 

\begin{document}


\begin{minipage}{0.5\textwidth}
    \raggedright
    {\Large \textbf{ABDUL SABOOR}} \\ Department of Physics and Astronomy, University of Delaware, Newark, DE 19716 
\end{minipage}
\hfill
\begin{minipage}{0.3\textwidth}
    \raggedright
    \faGithub \quad \href{https://github.com/asaboor-gh}{asaboor-gh}\\
    \faLinkedin \quad \href{https://linkedin.com/in/asaboor-in}{asaboor-in} \\
    \faPhone \quad \href{tel:+13027227047}{(302) 722-7047} \\
    \faEnvelope \quad \href{mailto:asaboor@udel.edu}{asaboor@udel.edu}
\end{minipage}
\vspace{2mm}
\hrule
\vspace{5mm} \noindent I am a graduate student working on semiconductor material modeling, electronic structure tuning in III-V semiconductors, and investigating strain effects in materials using density functional theory (DFT). My research involves extensive use of Python programming, including the development of open source data analysis packages for myself and the research community.

\section{Education}
\begin{tabular}{p{3cm} p{12cm}}
    \textbf{2025(Tentative)} & Ph.D. in Physics, University of Delaware \\
    \textbf{2017} & M.Phil. in Physics, Quaid-i-Azam University \\
    \textbf{2015} & M.Sc. in Physics, Quaid-i-Azam University \\
    \textbf{2012} & B.Sc. in Mathematics \& Physics, University of Azad Jammu \& Kashmir \\
\end{tabular}

\section{Research Experience}

\begin{itemize}
\item Investigated band-gap engineering in III-V alloys incorporating Bi for mid-infrared applications.
\item Analyzed rare-earth monopnictide nanoparticles embedded in bismide III-V alloys.

\item Quantified epitaxial strain effects on III-V materials and transition-metal dichalcogenides.
\item Examined metal oxidation in IrO$_2$ using density functional theory.
\end{itemize}

\section{Technical \& Computational Skills}
\begin{itemize}
\item Developed \href{https://github.com/asaboor-gh/ipyvasp}{ipyvasp}, a Python package for the analysis of \href{https://en.wikipedia.org/wiki/Vienna_Ab_initio_Simulation_Package}{VASP} calculations in Jupyter.
\item Authored \href{https://github.com/asaboor-gh/ipyslides}{ipyslides}, a package for building presentations using Python purely.
\item Proficient in Python, PowerShell, MATLAB, and Mathematica; and expanding expertise in Julia.
\item Extensive experience with computational simulations and data analysis.
\end{itemize}

\section{Publications}
\begin{itemize}
    \item S. Nair, et al., "Engineering metal oxidation using epitaxial strain," \href{https://www.nature.com/articles/s41565-023-01397-0}{\textit{Nat. Nanotechnol.}, (2023)}
    \item A. Saboor, S. Khalid, A. Janotti, "Band-gap reduction and band alignments of dilute bismide III-V alloys," \href{https://arxiv.org/abs/2411.19257}{\textit{arXiv:2411.19257} [cond-mat]}

\end{itemize}

\section{Conference Participations}
\begin{itemize}
\item The 67$^{th}$ Electronic Materials Conference, Duke University NC 2025
\item PyCon US, Pittsburgh 2025
\item The Franklin Institute Awards Symposium, Temple University, 2025
\item American Physical Society (APS) March Meetings, Las Vegas, 2023 and Minneapolis, 2024
\item Workshop for density functional theory, Temple University, 2019
\end{itemize}

\section{Honors \& Awards}
\begin{itemize}
\item University-funded presentation at the APS March Meeting on III-V semiconductor alloys. (2023)
\item Awarded prestigious World Federation of Scientists (WFS) scholarship, Switzerland (2016)
\item Offered competitive scholarship for doctoral studies at \textit{Universidad Nacional Autónoma de México} (UNAM), Mexico (2015)
\end{itemize}

\section{References}
\textbf{Prof. Anderson Janotti} \\
Department of Material Science and Engineering, University of Delaware \\
Email: \href{mailto:janotti@udel.edu}{janotti@udel.edu}

\end{document}
